\section{Questão 1}
Responda as seguintes perguntas usando o dataset fornecido:

1. Visualize os dados (em 2 ou 3 dimensões) para entender a estrutura dos dados.
Explique o que você fez para visualizar as figuras. Descreva sua abordagem para visualizar os dados e quais insights podem ser obtidos do gráfico, se algum.

2. Ajuste uma mistura de gaussianas com 4 componentes ao conjunto de dados. Calcule TODOS os parâmetros necessários para o modelo. Explique todas as etapas. Forneça detalhes de como você determina os parâmetros de melhor ajuste para cada modelo de mistura e descreva o processo de ajuste do modelo.

3. Suponha que as classes 1 e 2 sejam uma mesma classe. Ajuste uma mistura de gaussianas com 3 componentes ao conjunto de dados.

4. Suponha que as classes 1, 2 e 3 sejam uma mesma classe. Ajuste uma mistura de gaussianas com 2 componentes ao conjunto de dados.

5. Mostre como estimar qual dos 3 modelos (2, 3 ou 4 gaussianos) melhor representa o conjunto de dados original, ignorando as classes do modelo. Justifique como chegou ao resultado, usando técnicas de avaliação de modelos como AIC, BIC ou outro critério. Estude e explique o que é AIC e BIC.

6. Gere novas amostras com base no melhor modelo pelo seu critério. Plote os resultados. Visualize as amostras geradas e compare-as com o conjunto de dados original.
