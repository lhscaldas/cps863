\section*{Questão 2}

É dado um novo dataset com vários dados faltantes.

\begin{enumerate}


\item Descreva as etapas para executar a imputação de dados para as amostras incompletas. Explique sua abordagem. Descreva (mostre a matemática) que você usou para preencher os valores ausentes.

\begin{tcolorbox}[colback=white, colframe=black, title=Previsão da Feature Faltante]
    \textbf{Passo a Passo para Calcular \(\hat{x}_1\)}

    1. Definição da Expectativa Condicional:
    A previsão de \(x_1\) dada \(x_2\), \(x_3\) e \(y\) é expressa como a expectativa condicional:
    \[
    \hat{x}_1 = \mathbb{E}[x_1 | x_2, x_3, y]
    \]

    2.*Distribuição Condicional:
    Utilizamos a distribuição condicional em termos da distribuição conjunta:
    \[
    p(x_1 | x_2, x_3, y) = \frac{p(x_1, x_2, x_3 | y)}{p(x_2, x_3 | y)}
    \]

    3. Distribuição Conjunta:
    A distribuição conjunta pode ser modelada como uma mistura de gaussianas:
    \[
    p(x_1, x_2, x_3 | y) = \sum_{k=1}^{K} \pi_k \mathcal{N}([x_1, x_2, x_3] | \mu_k^y, \Sigma_k^y)
    \]

    4. Expectativa Condicional:
    A previsão condicional \(\hat{x}_1\) pode ser expressa como:
    \[
    \hat{x}_1 = \frac{1}{p(x_2, x_3 | y)} \sum_{k=1}^{K} \pi_k \mathbb{E}[x_1 | x_2, x_3, k]
    \]

    5. Fórmula Final da Previsão:
    A previsão \(\hat{x}_1\) resulta na fórmula:
    \[
    \hat{x}_1 = \sum_{k=1}^{K} \pi_k \left( \mu_k^y + \Sigma_{12} \Sigma_{kk}^{-1}(x_{\text{known}} - \mu_{k,\text{known}}) \right)
    \]

    onde:
    \begin{itemize}
        \item \(\Sigma_{12}\) é a covariância entre \(x_1\) e as features conhecidas \(x_2\) e \(x_3\).
        \item \(\Sigma_{kk}\) é a covariância entre as features conhecidas.
    \end{itemize}

    

\end{tcolorbox}


\begin{tcolorbox}[colback=white, colframe=black, title=Reposta(continuação):]

    \textbf{Analogamente, o cálculo de \(\hat{x}_2\) e \(\hat{x}_3\) é feito de forma similar.}

    \[
    \hat{x}_2 = \sum_{k=1}^{K} \pi_k \left( \mu_k^y + \Sigma_{21} \Sigma_{kk}^{-1}(x_{\text{known}} - \mu_{k,\text{known}}) \right)
    \]

    \[
    \hat{x}_3 = \sum_{k=1}^{K} \pi_k \left( \mu_k^y + \Sigma_{31} \Sigma_{kk}^{-1}(x_{\text{known}} - \mu_{k,\text{known}}) \right)
    \]

\end{tcolorbox}



\item Execute os cálculos necessários para imputar os valores ausentes. Forneça os detalhes matemáticos e demonstre o processo de imputação de valores ausentes para o conjunto de dados fornecido.

\begin{tcolorbox}[colback=white, colframe=black, title=Resposta:]

    Para preencher os valores faltantes no conjunto de dados, foi utilizada a previsão condicional com mistura de gaussianas. As etapas realizadas são as seguintes:
    
    \begin{enumerate}
        \item Carregamento do Dataset: O conjunto de dados foi carregado utilizando a biblioteca Pandas, onde as colunas `feature 2' e `feature 3' foram convertidas para o tipo numérico, tratando valores não numéricos como NaN. Esta conversão é crucial para garantir que todos os dados sejam adequadamente interpretados.
    
        \item Identificação de Linhas Completas e Incompletas: Foram separadas as linhas que contêm dados completos (sem NaN) e as que possuem pelo menos um valor ausente. As linhas completas foram usadas para ajustar o modelo de mistura de gaussianas.
    
        \item Ajuste do Modelo de Mistura de Gaussianas: Utilizando as linhas completas, o modelo foi ajustado para estimar os parâmetros necessários para prever os valores ausentes. A função \texttt{ajustar\_mistura\_gaussianas} foi utilizada para essa tarefa.
    
        \item Preenchimento dos Valores Faltantes: Para cada linha com dados faltantes, foi identificada a coluna com valor ausente. Os valores conhecidos da linha foram utilizados para prever o valor faltante, usando a função \texttt{calcular\_previsao\_condicional} com base nos resultados do modelo ajustado. O valor previsto foi então preenchido no DataFrame. As linhas que foram preenchidas podem ser observadas na tabela \ref{tab:linhas_preenchidas}.     
    \end{enumerate}
    
\end{tcolorbox}


\end{enumerate}
    
    
% \begin{longtable}[H]
%     \centering
%     \begin{tabular}{|c|c|c|c|}
%         \hline
%         \textbf{feature 1} & \textbf{feature 2} & \textbf{feature 3} & \textbf{class label} \\
%         \hline
%         2.839190300459528 & 9.56464817530692 & 7.750530975334812 & 3 \\
%         4.492978036403656 & 4.595353306087147 & 0.791578947368421 & 2 \\
%         5.8733673095703125 & 7.17497368316246 & 0.8905263157894735 & 3 \\
%         6.9206104800105095 & 2.2171797008249507 & 7.46922239639611 & 3 \\
%         6.694144755601882 & 4.938487235652284 & 0.8905263157894735 & 3 \\
%         6.708944767713547 & 2.109366426414244 & 7.106021649267843 & 3 \\
%         1.6811296929918367 & 4.2475351285698 & 4.79874519211732 & 2 \\
%         9.936176501214504 & 2.04472438940024 & 10.01282561819705 & 1 \\
%         6.315349519252777 & 1.790139209072315 & 6.030610810704607 & 3 \\
%         1.748088993363427 & 5.888952282607289 & 7.447549000035647 & 3 \\
%         1.929783437874051 & 9.449970443197673 & 10.207371287821273 & 1 \\
%         3.7911490201950073 & 2.3460556913090187 & 5.927534326445659 & 2 \\
%         2.0301167494036134 & 9.941293360481612 & 9.911052952029722 & 1 \\
%         4.348752379417419 & 1.713019560182466 & 4.328107931312082 & 2 \\
%         1.9578048353556075 & 9.587188626741376 & 10.931736145893224 & 1 \\
%         2.441918847473664 & 8.22632235851057 & 8.098441316210339 & 3 \\
%         2.151514055226405 & 5.436006256556076 & 6.359691261031645 & 2 \\
%         4.972970107570291 & 2.5725555649798286 & 6.499807943433078 & 2 \\
%         1.9677930111819368 & 4.971817448465 & 4.442982404013302 & 2 \\
%         6.072207152843475 & 1.5208369022000923 & 5.123386727269814 & 3 \\
%         3.144804000854492 & 1.3456178903579712 & 0.4126315789473684 & 4 \\
%         7.804545402526855 & 6.086988417393712 & 0.8905263157894735 & 3 \\
%         1.9602810542679305 & 4.952837770091846 & 5.28833540581244 & 2 \\
%         7.74591988325119 & 8.084379324586928 & 0.8905263157894735 & 3 \\
%         1.6114837256483783 & 4.071567923845636 & 5.165193804172858 & 2 \\
%         6.761785760521889 & 2.0930729759149465 & 7.051132365670919 & 3 \\
%         2.6984834671020503 & 7.244964335395697 & 0.791578947368421 & 2 \\
%         5.453582972288132 & 1.5478840975311094 & 3.91087737408126 & 2 \\
%         2.0641982679167863 & 5.2153945598961355 & 5.62621040266761 & 2 \\
%         3.691768765449524 & 2.191045305439699 & 5.535885745126899 & 2 \\
%         2.2119334086921807 & 7.451548717225432 & 6.228367138081465 & 3 \\
%         1.8495722269951174 & 6.230828424274331 & 7.427046872055546 & 3 \\
%         5.461026310920715 & 2.0015197729370575 & 5.057031341197352 & 2 \\
%         3.5419803857803345 & 2.4048383989315463 & 6.076054465385556 & 2 \\
%         5.071092829108238 & 4.28863785920351 & 0.791578947368421 & 2 \\
%         6.691092014312744 & 1.631090555329608 & 4.121106456284914 & 2 \\
%         1.9677491884476326 & 6.628942301507982 & 6.117473010913241 & 3 \\
%         7.624812126159668 & 6.561308513552811 & 0.8905263157894735 & 3 \\
%         6.777165666222572 & 1.7062844042569676 & 5.748121220014607 & 3 \\
%         2.684110058764383 & 6.781661052729159 & 4.309300230321324 & 2 \\
%         4.048401355743408 & 0.18524008267804193 & 1.795694679021835 & 4 \\
%         8.806843280792236 & 9.267883697468749 & 0.20421052631578948 & 1 \\
%         4.558074802160263 & 2.2311184042384125 & 5.637134266027904 & 2 \\
%         5.613296031951904 & 2.0494108785785947 & 5.178032804919321 & 2 \\
%         2.3574528075703287 & 5.9563302318931175 & 6.079464032391152 & 2 \\
%         5.156202197074889 & 2.6747955473839182 & 6.758127047911496 & 2 \\
%         9.117786347866058 & 1.9714560538475097 & 9.65403737708832 & 1 \\
%         6.741346657276154 & 7.449426608107276 & 0.8905263157894735 & 3 \\
%         3.685867547988892 & 6.881593746400178 & 0.791578947368421 & 2 \\
%         5.406050622463225 & 1.8005535212720136 & 4.549270864915992 & 2 \\
%         \hline
%     \end{tabular}
%     \caption{Conjunto de dados com valores faltantes preenchidos (apenas as linhas que foram preenchidas).}
%     \label{tab:linhas_preenchidas}
% \end{longtable}



\begin{longtable}{|c|c|c|c|}
    \hline
    \textbf{feature 1} & \textbf{feature 2} & \textbf{feature 3} & \textbf{class label} \\
    \hline
        2.839190300459528 & 9.56464817530692 & 7.750530975334812 & 3 \\
        4.492978036403656 & 4.595353306087147 & 0.791578947368421 & 2 \\
        5.8733673095703125 & 7.17497368316246 & 0.8905263157894735 & 3 \\
        6.9206104800105095 & 2.2171797008249507 & 7.46922239639611 & 3 \\
        6.694144755601882 & 4.938487235652284 & 0.8905263157894735 & 3 \\
        6.708944767713547 & 2.109366426414244 & 7.106021649267843 & 3 \\
        1.6811296929918367 & 4.2475351285698 & 4.79874519211732 & 2 \\
        9.936176501214504 & 2.04472438940024 & 10.01282561819705 & 1 \\
        6.315349519252777 & 1.790139209072315 & 6.030610810704607 & 3 \\
        1.748088993363427 & 5.888952282607289 & 7.447549000035647 & 3 \\
        1.929783437874051 & 9.449970443197673 & 10.207371287821273 & 1 \\
        3.7911490201950073 & 2.3460556913090187 & 5.927534326445659 & 2 \\
        2.0301167494036134 & 9.941293360481612 & 9.911052952029722 & 1 \\
        4.348752379417419 & 1.713019560182466 & 4.328107931312082 & 2 \\
        1.9578048353556075 & 9.587188626741376 & 10.931736145893224 & 1 \\
        2.441918847473664 & 8.22632235851057 & 8.098441316210339 & 3 \\
        2.151514055226405 & 5.436006256556076 & 6.359691261031645 & 2 \\
        4.972970107570291 & 2.5725555649798286 & 6.499807943433078 & 2 \\
        1.9677930111819368 & 4.971817448465 & 4.442982404013302 & 2 \\
        6.072207152843475 & 1.5208369022000923 & 5.123386727269814 & 3 \\
        3.144804000854492 & 1.3456178903579712 & 0.4126315789473684 & 4 \\
        7.804545402526855 & 6.086988417393712 & 0.8905263157894735 & 3 \\
        1.9602810542679305 & 4.952837770091846 & 5.28833540581244 & 2 \\
        7.74591988325119 & 8.084379324586928 & 0.8905263157894735 & 3 \\
        1.6114837256483783 & 4.071567923845636 & 5.165193804172858 & 2 \\
        6.761785760521889 & 2.0930729759149465 & 7.051132365670919 & 3 \\
        2.6984834671020503 & 7.244964335395697 & 0.791578947368421 & 2 \\
        5.453582972288132 & 1.5478840975311094 & 3.91087737408126 & 2 \\
        2.0641982679167863 & 5.2153945598961355 & 5.62621040266761 & 2 \\
        3.691768765449524 & 2.191045305439699 & 5.535885745126899 & 2 \\
        2.2119334086921807 & 7.451548717225432 & 6.228367138081465 & 3 \\
        1.8495722269951174 & 6.230828424274331 & 7.427046872055546 & 3 \\
        5.461026310920715 & 2.0015197729370575 & 5.057031341197352 & 2 \\
        3.5419803857803345 & 2.4048383989315463 & 6.076054465385556 & 2 \\
        5.071092829108238 & 4.28863785920351 & 0.791578947368421 & 2 \\
        6.691092014312744 & 1.631090555329608 & 4.121106456284914 & 2 \\
        1.9677491884476326 & 6.628942301507982 & 6.117473010913241 & 3 \\
        7.624812126159668 & 6.561308513552811 & 0.8905263157894735 & 3 \\
        6.777165666222572 & 1.7062844042569676 & 5.748121220014607 & 3 \\
        2.684110058764383 & 6.781661052729159 & 4.309300230321324 & 2 \\
        4.048401355743408 & 0.18524008267804193 & 1.795694679021835 & 4 \\
        8.806843280792236 & 9.267883697468749 & 0.20421052631578948 & 1 \\
        4.558074802160263 & 2.2311184042384125 & 5.637134266027904 & 2 \\
        5.613296031951904 & 2.0494108785785947 & 5.178032804919321 & 2 \\
        2.3574528075703287 & 5.9563302318931175 & 6.079464032391152 & 2 \\
        5.156202197074889 & 2.6747955473839182 & 6.758127047911496 & 2 \\
        9.117786347866058 & 1.9714560538475097 & 9.65403737708832 & 1 \\
        6.741346657276154 & 7.449426608107276 & 0.8905263157894735 & 3 \\
        3.685867547988892 & 6.881593746400178 & 0.791578947368421 & 2 \\
        5.406050622463225 & 1.8005535212720136 & 4.549270864915992 & 2 \\
        \hline
        \caption{Conjunto de dados com valores faltantes preenchidos (apenas as linhas que foram preenchidas).}
        \label{tab:linhas_preenchidas}
\end{longtable}

