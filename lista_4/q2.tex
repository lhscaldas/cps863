\section*{Questão 2}

Um robô pode se mover pelos quadrados da figura. A cada unidade de tempo, o robô tenta se mover para um dos quadrados adjacentes ao que ocupa, escolhendo aleatoriamente uma dentre as ações: (a) mover para Norte; (b) mover para Sul; (c) mover para Leste; (d) mover para Oeste. Caso o robô colida com uma parede, ele permanece na mesma posição até uma nova tentativa.

\begin{itemize}
    \item (1) Construa a cadeia de Markov que representa o movimento do robô pelo ambiente, e mostre a matriz de transição de estados do sistema. Indique claramente os estados do sistema e exemplifique as probabilidades de transição.
    \item (2) Suponha que o robô inicia sua caminhada no quadrado (2,1). Qual a probabilidade de estar no quadrado (1,3) vinte minutos após o início?
    \item (3) O robô está caminhando há muito tempo \( t \to \infty \). Onde você apostaria que o robô se encontra?
    \item (4) Quantos minutos em média leva para o robô retornar ao ponto de partida?
\end{itemize}