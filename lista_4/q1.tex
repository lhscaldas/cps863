\section*{Questão 1}

O objetivo desta questão é adquirir experiência com o algoritmo Expectation Maximization (EM). Para isso, você deve implementar sua solução, que deverá estar bem explicada. A implementação pode ser feita facilmente em Octave, R, Python, etc.

\begin{itemize}
    \item (a) Quantos parâmetros deverão ser estimados para o modelo Naive Bayes a ser construído?
    \item (b) Explique as equações usadas para resolver o problema.
    \item (c) Baseado no item anterior, explique sua implementação, incluindo as escolhas para a inicialização do código.
    \item (d) Quantas iterações foram necessárias para resolver o problema? Qual o teste de parada utilizado?
    \item (e) Quais os valores dos parâmetros encontrados? Quantos usuários foram alocados a cada uma das duas classes?
    \item (f) O resultado da clusterização fez algum sentido? Explique e justifique sua resposta.
    \item (g) Qual a probabilidade do i-ésimo usuário ser alocado ao cluster 1? Explique sua resposta de forma genérica e escolha um dos 1000 usuários para exemplificar.
    \item (h) Um usuário que votou 1,2,3,4,2 deveria ser alocado ao cluster 1 ou 2? Justifique.
    \item (i) Como você classificaria um usuário que votou 3,2,?,2,3? Isto é, o usuário não votou na feature número 3. Explique.
\end{itemize}
