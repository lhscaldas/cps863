\section*{Questão 2 - Para exercitar o EM mais uma vez}

Nesta tarefa, você usará o algoritmo Expectation-Maximization (EM) para inferir nota de filmes em um conjunto de dados. As notas são de 0.0 - 10.0 com uma casa decimal. O conjunto de dados contém as notas de clientes para quatro filmes de diferentes categorias (Sci-Fi e Romance). Os clientes são divididos em três classes com base em suas preferências, mas também é desconhecida a classe do cliente.

\begin{enumerate}
    \item \textbf{Explique as equações usadas para resolver o problema.}
    \item \textbf{Baseado no item anterior, explique a sua implementação, incluindo as suas escolhas para a inicialização do código.}
    \item \textbf{Quantas iterações foram necessárias para resolver o problema? Qual o teste de parada utilizado?}
    \item \textbf{Quais os valores dos parâmetros encontrados? Quantos usuários foram alocados a cada uma das duas classes?}
    \item \textbf{O resultado da clusterização fez algum sentido? Explique e justifique a sua resposta.}
    \item \textbf{Qual a probabilidade do i-ésimo cliente ser um cliente que gosta mais de Sci-Fi? Explique sua resposta de forma genérica e escolha um dos 1000 usuários para exemplificar.}
\end{enumerate}


