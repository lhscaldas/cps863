\documentclass[12 pt]{article}
\usepackage[utf8]{inputenc}
\usepackage{matlab-prettifier}
\usepackage[portuguese]{babel}
\usepackage{indentfirst}
\usepackage{graphicx}
\usepackage{float}
\usepackage{subcaption}
\usepackage[font=small,labelfont=bf]{caption}
\definecolor{mygreen}{RGB}{28,172,0} % color values Red, Green, Blue
\definecolor{myyellow}{rgb}{1.0, 1.0, 0.8}
\usepackage{mathtools}
\usepackage{multirow}
\usepackage{comment}
\usepackage{xcolor}
\usepackage{colortbl}
\usepackage[normalem]{ulem}               % to striketrhourhg text
\usepackage{amsmath}
\usepackage{amsfonts}
\usepackage{hyperref}
\usepackage{tcolorbox}
\newcommand\redout{\bgroup\markoverwith
{\textcolor{red}{\rule[0.5ex]{2pt}{0.8pt}}}\ULon}
\renewcommand{\lstlistingname}{Código}% Listing -> Algorithm
\renewcommand{\lstlistlistingname}{Lista de \lstlistingname s}% List of Listings -> List of Algorithms

\usepackage[top=3cm,left=2cm,bottom=2cm, right=2cm]{geometry}
\usepackage{tikz}
\usetikzlibrary{decorations.pathreplacing}


% Configuração para destacar a sintaxe do Python
\lstset{ 
    language=Python,                     % A linguagem do código
    backgroundcolor=\color{myyellow}, % A cor do fundo 
    basicstyle=\ttfamily\footnotesize,   % O estilo do texto básico
    keywordstyle=\color{blue},           % Cor das palavras-chave
    stringstyle=\color{red},             % Cor das strings
    commentstyle=\color{mygreen},          % Cor dos comentários
    numbers=left,                        % Números das linhas à esquerda
    numberstyle=\tiny\color{gray},       % Estilo dos números das linhas
    stepnumber=1,                        % Número de linhas entre os números das linhas
    frame=single,                        % Moldura ao redor do código
    breaklines=true,                     % Quebra automática das linhas longas
    captionpos=t,                        % Posição da legenda
    showstringspaces=false               % Não mostra espaços em branco nas strings
    extendedchars=true,
    literate={º}{{${ }^{\underline{o}}$}}1 {á}{{\'a}}1 {à}{{\`a}}1 {ã}{{\~a}}1 {é}{{\'e}}1 {É}{{\'E}}1 {ê}{{\^e}}1 {ë}{{\"e}}1 {í}{{\'i}}1 {ç}{{\c{c}}}1 {Ç}{{\c{C}}}1 {õ}{{\~o}}1 {ó}{{\'o}}1 {ô}{{\^o}}1 {ú}{{\'u}}1 {â}{{\^a}}1 {~}{{$\sim$}}1
}


\title{%
\textbf{\huge Universidade Federal do Rio de Janeiro} \par
\textbf{\LARGE Instituto Alberto Luiz Coimbra de Pós-Graduação e Pesquisa de Engenharia} \par

\includegraphics[width=8cm]{COPPE UFRJ.png} \par

\textbf{Programa de Engenharia de Sistemas e Computação} \par

CPS863 - Aprendizado de Máquina  \par

Prof. Dr. Edmundo de Souza e Silva (PESC/COPPE/UFRJ)\par

\vspace{1\baselineskip}
\textbf{\textit{Lista de Exercícios 1a}} \par
}

\author{Luiz Henrique Souza Caldas\\email: lhscaldas@cos.ufrj.br}

\date{\today}

\begin{document}
\maketitle


\section*{Questão 1}
\textbf{(Recordação)}

Uma caixa contém três moedas: duas são normais e uma moeda falsa com duas caras (P(Ca)=1). Se você pegar uma moeda da caixa e jogá-la, qual a probabilidade de sair cara? Se você pegar uma moeda da caixa e jogá-la, e sair cara, qual a probabilidade de ser a moeda falsa?

\section*{Questão 2}
\textbf{(Material introdutório)}

Uma urna $UA$ tem $N = 1000$ bolas sendo 25\% delas azuis e o restante pretas. Uma outra urna $UB$ também contém $N = 1000$ bolas, mas apenas 10\% delas são azuis (e o restante pretas). As urnas são idênticas externamente, exceto por uma marcação, $UA$, $UB$, que permite a identificação de cada uma. Entretanto, essa identificação está na parte inferior das urnas, de forma que não é possível visualizar o rótulo, exceto se a urna for levantada.

\begin{itemize}
    \item João tira (de olhos vendados) 2 bolas azuis de uma das urnas. Você vai ter que adivinhar a urna escolhida. Se a probabilidade de João escolher uma das urnas for a mesma, qual a aposta que você fará? Note que, para fazer a aposta, você precisa determinar qual a probabilidade das bolas serem provenientes da urna $UA$. Você tem confiança na sua aposta? Por que?
    \item Um amigo seu diz que João sabe a posição das urnas e escolhe a urna $UA$ com probabilidade $0.15$. Sua aposta mudaria? Você teria confiança na sua aposta? Justifique a resposta.
\end{itemize}

\section*{Questão 3}
Considere um dataset cujas amostras são obtidas independentemente a partir de uma distribuição discreta uniforme $U(1, 5)$. Considere um dataset com as seguintes amostras: $\{2, 2, 4, 3, 2\}$.
\begin{enumerate}
    \item Qual a verossimilhança (likelihood) de observar essas amostras?
    \item E o log-likelihood?
\end{enumerate}

\section*{Questão 4}
Assuma que você tem uma moeda viciada tal que com probabilidade $p$ você obtém caras (H) e $(1 - p)$ coroas (T). Você joga a moeda $N$ vezes e obtém $N_H$ caras (e $N - N_H$ coroas, é claro).
\begin{enumerate}
    \item Obtenha a função de verossimilhança $L(\theta|D) = p(D|\theta)$ onde $\theta$ é o vetor de parâmetros. Qual é o valor de $p(D|\theta)$ se $D = \{HHT HT T HT T T\}$ e $p = 0.2$? E se $p = 0.6$?
    \item Para $D$ dado no item acima, encontre $p$ que otimiza o log-likelihood. De maneira geral, encontre $p$ como uma função de $N$ e $N_H$ para qualquer conjunto $D$ dado.
\end{enumerate}

\section*{Questão 5}
Suponha agora que suas amostras são obtidas ou de uma distribuição discreta $U(1, 5)$ ou a partir de um dado (seis faces), sendo que todas as amostras são obtidas da mesma distribuição. Suponha que a probabilidade das amostras serem obtidas do dado é igual a $p$. Considere o conjunto de dados $\{2, 2, 4, 3, 2\}$, e seja $p = 0.7$.
\begin{enumerate}
    \item Qual a likelihood das amostras serem retiradas a partir: (a) do dado se seis faces, ou (b) de uma $U(1, 5)$ discreta?
    \item Qual a distribuição posterior?
    \item Uma vez que o dataset acima foi observado, qual a probabilidade de se retirar o número 5?
    \item Uma vez que o dataset acima foi observado, qual a probabilidade de se retirar o número 6?
    \item Qual o MLE?
    \item Qual o MAP?
    \item Caso $p = 0.5$, quais das respostas acima mudam de valor? Explique.
\end{enumerate}

\section*{Questão 6}
Suponha agora que suas amostras são obtidas ou de uma distribuição discreta $U(1, 5)$ ou a partir de um dado (seis faces) com probabilidade $(1 - p)$ e $p$, respectivamente. Entretanto, nesta questão, a sequência pode conter amostras de ambas distribuições (mistura de distribuições). Considere o mesmo conjunto de dados $\{2, 2, 4, 3, 2\}$, e seja $p = 0.7$.
\begin{enumerate}
    \item Qual a likelihood de observar essas amostras?
    \item Uma vez que o dataset acima foi observado, qual a probabilidade de se retirar o número 5?
    \item Uma vez que o dataset acima foi observado, qual a probabilidade de se retirar o número 6?
    \item Qual a probabilidade de \textbf{todas} as amostras serem retiradas a partir: (a) do dado se seis faces, ou (b) de uma $U(1, 5)$ discreta?
    \item Calcule a função de likelihood para as amostras em função de $p$, o log likelihood e obtenha o valor de $p$ que melhor explica o conjunto de dados. Comente a sua resposta.
    \item Repita o item anterior, supondo que o conjunto de dados tem cardinalidade 20 e apenas uma única amostra tenha valor igual a 6.
\end{enumerate}


\end{document}